% !TEX program = xelatex
% 使用 texlive完整编译:
% xelatex -> bibtex -> xelatex -> xelatex
% 中文书籍 LaTeX 模板

\documentclass[UTF8,openany,twoside,12pt]{ctexbook}
%\documentclass[UTF8,openany,twoside,zihao=-4]{ctexbook}
%\documentclass[UTF8,twoside,12pt]{book}

%----- package for template -----
\usepackage{amsmath,amsthm,amssymb,amsfonts}
\usepackage{mathrsfs,bm}
%\usepackage[notcite,notref]{showkeys}
\usepackage{geometry}
\geometry{left=2.5cm,right=2.5cm,top=3.0cm,bottom=3.0cm}
%\geometry{left=1in,right=1in,top=1in,bottom=1in}
%\geometry{left=2.5cm,right=2.1cm,top=1.7cm,bottom=2cm,includehead,includefoot}
%\geometry{b5paper,text={125mm,195mm},centering,left=1in,right=1in,top=1in,bottom=1in}
\usepackage{url,hyperref}
\hypersetup{colorlinks=true,linkcolor=black,citecolor=black} % 去掉目录红框
\usepackage[english]{babel}
\usepackage{imakeidx}
\usepackage{color,xcolor}
\usepackage{graphicx}
\usepackage{epsfig}
\usepackage{tabularx,array}
\usepackage{longtable}
\usepackage{booktabs}
\usepackage{multirow}
\usepackage{multicol}
\usepackage{fancybox}
\usepackage{makecell}
\usepackage{xstring}
%\usepackage[english]{babel}
\usepackage{listings}
\usepackage{titletoc}
\usepackage{titlesec}
\usepackage{mathtools}
\usepackage{float}
\usepackage{bookmark}
\usepackage[numbers]{natbib}
%\def\bibfont{\small}  % 修改参考文献字体

%\usepackage[title,titletoc]{appendix}
%\renewcommand{\appendixname}{附录}

% --- 设置英文字体 -----
%\usepackage{newtxtext}  % for text fonts
%\setmainfont{Times New Roman}
%\setmonofont{Courier New}
%\setsansfont{Arial}

% --- 设置数学字体 -----
% \usepackage{newtxmath}
% \usepackage{mathptmx}

% --- 直接插入 pdf 文件 ----
% \usepackage{pdfpages}

% --- 算法宏包及设置 ---
\usepackage{algorithm}
\usepackage{algpseudocode}
\floatname{algorithm}{算法}
\algrenewcommand\algorithmicrequire{\textbf{输入:}}
\algrenewcommand\algorithmicensure{\textbf{输出:}}

% ---- 定义列表项的样式 -----
\usepackage{enumitem}
\setlist{nolistsep}
% \setlength{\itemsep}{3pt plus1pt minus2pt}


\usepackage{indentfirst}                % 首行缩进宏包
%\usepackage[perpage,symbol]{footmisc}  % 脚注控制
\usepackage{fancyhdr,fancyref}          % 页眉和页脚的相关定义

\usepackage[cleardoublepage=plain]{scrextend}
\pagestyle{fancy}
\fancyhf{}  %清除以前对页眉页脚的设置
\fancyhead[RO,LE]{\fangsong \zihao{5} ~\thepage~}
\fancyhead[LO]{\fangsong \zihao{5} \rightmark}
\fancyhead[RE]{\fangsong \zihao{5} \leftmark}
%\fancyhead[LO,RE]{\fangsong \zihao{5} \leftmark}
\lfoot{}
\cfoot{}
\rfoot{}


\theoremstyle{plain}
\newtheorem{definition}{定义}[chapter]
\newtheorem{proposition}{命题}[chapter]
\newtheorem{lemma}{引理}[chapter]
\newtheorem{theorem}{定理}[chapter]
\newtheorem{example}{例}[chapter]
\newtheorem{corollary}{推论}[chapter]
\newtheorem{remark}{注}[chapter]
\newtheorem{exercise}{练习}[chapter]
\newtheorem{assumption}{假设}[chapter]
\newtheorem{axiom}{公理}[chapter]
\newtheorem{property}{性质}[chapter]
\newtheorem{conjecture}{猜想}[chapter]
\renewcommand{\proofname}{\bfseries 证明}

%% ----- 重新设置图表autoref -------
\renewcommand{\figureautorefname}{图}
\renewcommand{\tableautorefname}{表}

% --- 使用 tabularx库并定义新的左右中格式 ----
\newcolumntype{L}{X}
\newcolumntype{C}{>{\centering \arraybackslash}X}
\newcolumntype{R}{>{\raggedleft \arraybackslash}X}
\newcolumntype{P}[1]{>{\centering \arraybackslash}p{#1}}

%----- 设置章节标题格式 -----
%\renewcommand{\chaptername}{第\chinese{chapter}章}
\ctexset{chapter={
format ={\bfseries \filcenter \LARGE},
name={第,章},
number=\arabic{chapter},
beforeskip = {-3ex plus 0.2ex minus 0.2ex},
afterskip = {3ex plus 0.2ex minus 0.2ex},
},
%section={name=\S,number=\arabic{section}},
section={number=\arabic{chapter}.\arabic{section},
beforeskip = {1.2ex plus 0.2ex minus 0.2ex},
afterskip = {1.5ex plus 0.2ex minus 0.2ex},
},
subsection={number=\arabic{chapter}.\arabic{section}.\arabic{subsection},
beforeskip = {1ex plus 0.2ex minus 0.2ex},
afterskip = {1ex plus 0.2ex minus 0.2ex},
},
}

%\ctexset{
%chapter={name={第,章},number=\chinese{chapter}},
%section={name={{\bf\S}},number={\normalsize{\arabic{section}}}},
%subsection={number=\arabic{section}.\arabic{subsection}},
%}

%\hypersetup{
%	colorlinks = true,
%	linkcolor  = black,
%	citecolor = black
%} % 去掉目录红框

% ------------- 定义新的目录页面 ----------------
\usepackage{tocloft}
%\renewcommand{\cfttoctitlefont}{\hfill \heiti \chpzihao}
%\renewcommand{\cftchappresnum}{第}
%\renewcommand{\cftchapaftersnum}{章}
\renewcommand{\contentsname}{\hfill\LARGE 目~~录}
\renewcommand{\cftaftertoctitle}{\hfill}
%\renewcommand{\cftchapaftersnumb}{\hspace{28pt}}
%\setlength{\cftbeforetoctitleskip}{5pt}
%\setlength{\cftaftertoctitleskip}{24}
\setlength{\cftbeforechapskip}{12pt}
\setlength{\cftbeforesecskip}{3pt}
\renewcommand{\cftdot}{$\cdot$}
\renewcommand{\cftdotsep}{3}
\renewcommand{\cftchapdotsep}{\cftdotsep}
\renewcommand\cftchapleader{\cftdotfill{\cftchapdotsep}}
\renewcommand{\cftsecdotsep}{\cftdotsep}  % 设置Section引导点


\makeindex
%\bibliographystyle{plain}

\renewcommand{\baselinestretch}{1.35}        % 定义行距

% --- 自定义命令 -----
\newcommand{\CC}{\ensuremath{\mathbb{C}}}
\newcommand{\RR}{\ensuremath{\mathbb{R}}}
\newcommand{\A}{\mathcal{A}}
\newcommand{\ii}{\bm{\mathrm{i}}\,}  % 虚部
\newcommand{\md}{\mathrm{d}\,}
\newcommand{\bA}{\boldsymbol{A}}
\newcommand{\red}[1]{\textcolor{red}{#1}}


\graphicspath{{./figure/}{./figures/}{./image/}{./images/}}


\title{\bfseries LaTeX 书籍样例}
\author{\fangsong 某某某~~编著}
\date{2021年12月}


\begin{document}

\maketitle

\thispagestyle{empty}

\frontmatter


\chapter{序}

这部分是序言这部分是序言这部分是序言这部分是序言这部分是序言这部分是序言这部分是序言这部分是序言这部分是序言这部分是序言这部分是序言这部分是序言这部分是序言这部分是序言这部分是序言这部分是序言这部分是序言这部分是序言这部分是序言这部分是序言这部分是序言这部分是序言这部分是序言这部分是序言这部分是序言这部分是序言这部分是序言这部分是序言这部分是序言这部分是序言这部分是序言这部分是序言这部分是序言这部分是序言这部分是序言这部分是序言这部分是序言这部分

\chapter{前~~言}

这部分是前言这部分是前言这部分是前言这部分是前言这部分是前言这部分是前言这部分是前言这部分是前言这部分是前言这部分是前言这部分是前言这部分是前言这部分是前言这部分是前言这部分是前言这部分是前言这部分是前言这部分是前言这部分是前言这部分是前言这部分是前言这部分是前言这部分是前言这部分是前言这部分是前言这部分是前言这部分是前言这部分是前言这部分是前言这部分是前言这部分是前言这部分是前言这部分是前言这部分是前言这部分是前言这部分是前言这部分是前言这部分


\renewcommand\contentsname{目~~录}
\cleardoublepage
\phantomsection
\pdfbookmark[chapter]{\contentsname}{toc}
\tableofcontents

\mainmatter

%\part{总论}

\chapter{第一个章节标题}

\section{引言}

\subsection{研究背景}

这是小四号的正文字体, 行间距 1.35 倍.

通过空一行实现段落换行, 仅仅是回车并不会产生新的段落.

自定义了一个命令 \verb|\red{文字}| 可以\red{加红文字}, 可以在论文修改阶段方便标记.

这是一个引用的示例 \cite{Tadmor2012} 和 \cite{LiLiu1997,Adams2003,TreWei2014}.

这是一大段文字这是一大段文字中英文混排 Numerical Methods. 这是一大段文字这是一大段文字这是一大段文字这是一大段文字这是一大段文字这是一大段文字这是一大段文字这是一大段文字这是一大段文字这是一大段文字.

\subsection{列表的使用}

这是一个计数的列表.
\begin{enumerate}%[label={\rm (\arabic*)}]%\roman
	\item 第一项
		\begin{enumerate}
			\item 第一项中的第一项
			\item 第一项中的第二项
		\end{enumerate}
	\item 第二项
	\item 第三项
\end{enumerate}


这是一个不计数的列表.
\begin{itemize}%[label={$\bullet$}]
	\item 第一项
	\begin{itemize}
		\item 第一项中的第一项
		\item 第一项中的第二项
	\end{itemize}
	\item 第二项
	\item 第三项
\end{itemize}


%\subsection{文献引用}

%参考文献采用 BibLaTeX 的方式生成 (内容写在文件 \verb|mybib.bib| 中), 参考文献的样式可以选择 BibLaTeX 的标准样式: \verb|plain|、\verb|abbrv|、\verb|unsrt| 与 \verb|siam| 等.

%文献引用示例 \cite{LiLiu1997} 和 \cite{Adams2003,Shen1994}.



\subsection{数学公式}

数学公式的使用请参考公式手册 symbols-a4, 或者 《一份(不太)简短的 \LaTeX~2$\varepsilon$ 介绍》 (lshort-zh-cn).

自定义命令表示的几个数学符号 $\RR$, $\CC$, $\A$, $\ii$, $\md$, $\bA$.

在文中行内公式可以这么写: $a^2+b^2=c^2$, 这是勾股定理, 它还可以表示为 $c=\sqrt{a^2+b^2}$, 还可以让公式单独一段并且加上编号
\begin{equation}\label{eqn:trifun}
\sin^2{\theta}+\cos^2{\theta}=1.
\end{equation}
还可以通过添加标签在正文中引用公式, 如等式~\eqref{eqn:trifun} 或者 \ref{eqn:trifun}.

%读者可能阅读过其它手册或者资料, 知道 LaTeX 提供了 eqnarray 环境. 它按照等号左边—等号—等号右边呈三列对齐, 但等号周围的空隙过大, 加上公式编号等一些 bug, 目前已不推荐使用. (摘自 lshort-zh-cn)

多行公式常用 align 环境, 公式通过 \verb|&| 对齐. 分隔符通常放在等号左边:
\begin{align}
a & = b + c \\
& = d + e.
\end{align}

align 环境会给每行公式都编号. 我们仍然可以用 \verb|\notag| 或 \verb|\nonumber| 去掉某行的编号. 在以下的例子,
为了对齐等号, 我们将分隔符放在右侧, 并且此时需要在等号后添加一对括号 \verb|{}| 以产生正常的间距:
\begin{align}
a ={} & b + c \\
={} & d + e + f + g + h + i + j \notag \\
& + m + n + o \\
={} & p + q + r + s.
\end{align}
如果两列都要左对齐, 然后整体居中
\begin{align}
&a = b + 1 && e = c + f + g + 3 \\
&c = d + e + 2 && c = d + e + 2 \\
&e = c + f + g + 3 && a = b + 1
\end{align}
如果我们不需要按等号对齐, 只需罗列数个公式, gather 将是一个很好用的环境:
\begin{gather}
a = b + c \\
d = e + f + g \\
h + i = j + k \notag \\
l + m = n
\end{gather}

align 和 gather 有对应的不带编号的版本 align* 和 gather*.
对于 align、 gather、align* 与 gather* 等环境, 若添加命令 \verb|\allowdisplaybreaks| 后 (已添加), 公式可以跨页显示.

多个公式组在一起公用一个编号, 编号位于公式的居中位置, amsmath 宏包提供了诸如 aligned、gathered 等环境, 与 equation 环境套用.

这个公式使用 aligned 环境 (\textbf{推荐使用})
\begin{equation}\label{eqn:1}
\left\{\begin{aligned}
  &-\frac{\mathrm{d}^{2} u}{\mathrm{d} x^{2}}+\frac{\mathrm{d} u}{\mathrm{d} x}=\pi^{2} \sin (\pi x)+\pi \cos (\pi x),\quad x \in [0,1], \\
  &u(0)=0,\quad u(1)=0.
\end{aligned} \right.
\end{equation}

这个公式使用 array 环境
\begin{equation}\label{eqn:2}
\left\{\begin{array}{l}
\displaystyle
-\frac{\mathrm{d}^{2} u}{\mathrm{d} x^{2}}+\frac{\mathrm{d} u}{\mathrm{d} x}=\pi^{2} \sin (\pi x)+\pi \cos (\pi x),\quad x \in [0,1], \\[6pt]
u(0)=0,\quad u(1)=0.
\end{array} \right.
\end{equation}

aligned 与 equation 环境套用, 公式间距是自动调节的, 如果有分式, 分式也是行间显示. 如果用 array 与 equation 环境套用, 有时候需要手动调整公式行间距和行间显示.

\subsection{定理环境}

\begin{definition}
这是一个定义.
\end{definition}

\begin{proposition}
这是一个命题.
\end{proposition}

\begin{lemma}[Lemma]\label{lemma1}
这是一个引理.
\end{lemma}

\begin{theorem}[Theorem]
这是一个定理.
\end{theorem}
\begin{proof}[\normalfont\bfseries 证明\,:\nopunct]
这是证明环境.
\end{proof}

\begin{proposition}[Proposition]
这是一个命题.
\end{proposition}

\begin{lemma}\label{lemma-convergence} {\rm (\textit{参考文献}\cite{LiLiu1997})}
假设单步法具有 $p$ 阶精度, 且増量函数 $\varphi(x_{n}, u_{n}, h)$ 关于 $u$ 满足 {\rm Lipschitz} 条件
\begin{equation}\label{eqn:3}
|\varphi(x, u, h)-\varphi(x, \bar{u}, h)| \leqslant L_{\varphi}|u-\bar{u}|.
\end{equation}
\end{lemma}

\begin{theorem}\label{theorem-convergence}
假设单步法具有 $p$ 阶精度, 且増量函数 $\varphi(x_{n}, u_{n}, h)$ 关于 $u$ 满足 {\rm Lipschitz} 条件
\begin{equation}\label{eqn:4}
|\varphi(x, u, h)-\varphi(x, \bar{u}, h)| \leqslant L_{\varphi}|u-\bar{u}|.
\end{equation}
\end{theorem}
\begin{proof}[\normalfont\bfseries 证明\,:\nopunct]
由定理 \ref{lemma1} 和 \eqref{eqn:1} 式可以推出以上结论.
\end{proof}

\begin{corollary}\label{col-convergence}
假设单步法具有 $p$ 阶精度, 且増量函数 $\varphi(x_{n}, u_{n}, h)$ 关于 $u$ 满足 {\rm Lipschitz} 条件
\begin{equation}\label{eqn:5}
|\varphi(x, u, h)-\varphi(x, \bar{u}, h)| \leqslant L_{\varphi}|u-\bar{u}|.
\end{equation}
\end{corollary}


\begin{remark}\label{remark1}
这是一个 remark.
\end{remark}

\begin{example}
这是一个例子.
\end{example}


%
%\subsection{算法环境}
%
%如下是算法~\ref{alg:euclid}.
%\begin{algorithm}[H]
%    \small
%    \caption{Euclid's algorithm}\label{alg:euclid}
%    \begin{algorithmic}[1]
%        \Procedure{Euclid}{$a,b$}\Comment{The g.c.d. of a and b}
%        \State $r\gets a\bmod b$
%        \While{$r\not=0$}\Comment{We have the answer if r is 0}
%        \State $a\gets b$
%        \State $b\gets r$
%        \State $r\gets a\bmod b$
%        \EndWhile\label{euclidendwhile}
%        \State \Return $b$\Comment{The gcd is b}
%        \EndProcedure
%    \end{algorithmic}
%\end{algorithm}


%%%%%%%%%%%%%%%%%%%%%%%%%%%% 微分方程的数值方法  %%%%%%%%%%%%%%%%%%%%%%%

\section{微分方程的数值方法}

本章我们考虑具有以下微分方程:
\begin{equation}\label{eqn:pde}
\left\{\begin{aligned}
& L u=-\frac{\mathrm{d}^{2} u}{\mathrm{d} x^{2}}+\frac{\mathrm{d} u}{\mathrm{d} x}+q u=f, \quad a < x < b, \\
& u(a)=\alpha, \quad u(b)=\beta.
\end{aligned}\right.
\end{equation}
其中 $q, f$ 为 $[a,b]$ 上的连续函数, $q \geqslant 0$; $\alpha, \beta$ 为给定常数. 这是最简单的椭圆方程第一边值问题 .

问题 \eqref{eqn:pde} 存在唯一解 (引用示例参考文献 \cite{LiLiu1997}).


\subsection{有限差分方法}
在偏微分方程的数值解法中, 有限差分法数学概念直观, 推导自然, 是发展较早且比较成熟的数值方法. 由于计算机只能存储有限个数据和做有限次运算, 所以任何一种用计算机解题的方法, 都必须把连续问题 (微分方程的边值问題、初值问题等) 离散化, 最终化成有限形式的线性代数方程组.

\subsubsection{数值格式}
将区间 $[a,b]$ 分成 $N$ 等分, 分点为
\begin{equation*}
  x_{i}=a+i h \quad i=0,1, \cdots, N,
\end{equation*}
其中 $h=(b-a) / N$. 于是我们得到区间 $I=[a,b]$ 的一个网格剖分. $x_i$ 称为网格的节点, $h$ 称为步长.

数值格式:
\begin{equation*}
  L_{h} u_{i}=-\frac{u_{i+1}-2 u_{i}+u_{i-1}}{h^{2}}+\frac{u_{i+1}-u_{i-1}}{h}+q_{i} u_{i}=f_{i},\quad 1 \leqslant j \leqslant N-1.
\end{equation*}
其中  $q_{i}=q(x_{i}), f_{i}=f(x_{i})$.

以上差分方程对于 $i=1,2, \cdots, N-1$ 都成立, 加上边值条件 $u_{0}=\alpha, u_{N}=\beta$, 就得到关于 $u_i$ 的差分格式:
\begin{equation}\label{eqn:fdm}
\left\{\begin{aligned}
& L_{h} u_{i}=-\frac{u_{i+1}-2 u_{i}+u_{i-1}}{h^{2}}+\frac{u_{i+1}-u_{i-1}}{2h}+q_{i} u_{i}=f_{i}, \quad i=1,2, \cdots, N-1, \\
& u_{0}=\alpha, \quad u_{N}=\beta.
\end{aligned}\right.
\end{equation}

它的解 $u_i$ 是 $u(x)$ 在 $x=x_i$ 处的差分解.


\subsection{矩阵形式}

先定义向量 $\boldsymbol{u}$:
\begin{equation*}
  \boldsymbol{u}=(u_{1}, u_{2}, \cdots, u_{N-1})^{\mathrm{T}}.
\end{equation*}

差分格式可以写为矩阵形式:
\begin{equation*}
  \boldsymbol{A}\boldsymbol{u}=\boldsymbol{f}.
\end{equation*}
其中矩阵 $\boldsymbol{A}$、向量 $\boldsymbol{f}$ 的定义如下, 注意向量 $\boldsymbol{f}$ 的首尾元素已包含了 $x=a$ 和 $x=b$ 处的边界条件.
\begin{equation}\label{equ:matrix1}
\boldsymbol{A}=\begin{bmatrix}
\dfrac{2}{h^{2}}+q_{1} & \dfrac{1}{2h}-\dfrac{1}{h^{2}} &   &  &  \\[8pt]
 -\dfrac{1}{2h}-\dfrac{1}{h^{2}} & \dfrac{2}{h^{2}}+q_{2} & \dfrac{1}{2h}-\dfrac{1}{h^{2}}  & &  \\[8pt]
  &  &  &  &    \\
   &  \ddots  & \ddots  &  \ddots  &  \\[8pt]
   &  &  &  &    \\
  &   & -\dfrac{1}{2h}-\dfrac{1}{h^{2}} & \dfrac{2}{h^{2}}+q_{N-2}& \dfrac{1}{2h}-\dfrac{1}{h^{2}} \\[8pt]
  &  &  & -\dfrac{1}{2h}-\dfrac{1}{h^{2}} & \dfrac{2}{h^{2}}+q_{N-1}
\end{bmatrix}.
\end{equation}

上一个矩阵用了 \verb|bmatrix| 环境, 也可以使用 \verb|array| 环境.
\begin{equation}\label{equ:matrix2}
\boldsymbol{A}=\left[\begin{array}{cccccc}
\dfrac{2}{h^{2}}+q_{1} & \dfrac{1}{2h}-\dfrac{1}{h^{2}} &   &  &  \\[8pt]
 -\dfrac{1}{2h}-\dfrac{1}{h^{2}} & \dfrac{2}{h^{2}}+q_{2} & \dfrac{1}{2h}-\dfrac{1}{h^{2}}  & &  \\[8pt]
  &  &  &  &    \\
   &  \ddots  & \ddots  &  \ddots  &  \\[8pt]
   &  &  &  &    \\
  &   & -\dfrac{1}{2h}-\dfrac{1}{h^{2}} & \dfrac{2}{h^{2}}+q_{N-2}& \dfrac{1}{2h}-\dfrac{1}{h^{2}} \\[8pt]
  &  &  & -\dfrac{1}{2h}-\dfrac{1}{h^{2}} & \dfrac{2}{h^{2}}+q_{N-1}
\end{array}\right].
\end{equation}



%%%%%%%%%%%%%%%%%%%%%%%%%%%%%%%%%%%%%%%%%%%%%%%%%%%%%%%%%%%%%%%%%%%%%%%%%%

\backmatter  % 结束章节自动编号

%%%%%%%%%%%%%%%%%%%%%%%%%%%%%%%%%%%%%%%%%%%%%%%%%%%%%%%%%%%%%%%%%%%%%%%%%%


\appendix

\chapter{附录~A~~这是第一个附录}

\section*{A.1~~附录A的小节}

\clearpage
\section*{A.2~~附录A的小节}


\chapter{附录~B~~这是第二个附录}

\section*{B.1~~附录B的小节}


内容附录内容附录内容附录\index{内容}附录内容附录内容附录内容附录内容附录内容附录内容附录内容附录内容附录内容附录内容附录内容附录内容附录内容附录内容附录内容附录内容附录内容正文内容正文.内容附录内容附录内容附录内容附录内容附录内容附录内容附录内容附录内容附录内容附录内容附录内容附录内容附录内容附录内容附录内容附录内容附录内容文内容正文内容正文内容正文\index{内容}


%%%%%%%%%%%%%%%%%%%%%%%% 生成参考文献 %%%%%%%%%%%%%%%%%%%%%%%%%%%%%%%%%%%%%

\renewcommand{\bibname}{参考文献}

% 生成参考文献, 使用第一种请把第二种全部注释

% 第一种方式, 使用 bib 文件

%\nocite{*}  % 可以暂时显示全部参考文献, 包括未引用的

% 使用方法:\bibliography{参考文件1文件名, 参考文献2文件名, ...}
% 参考文献格式可选  plain, abbrv, unsrt, siam
%\bibliographystyle{plain}
\bibliographystyle{thuthesis-numeric}
\bibliography{mybib}


%----------------------------------------------------------------

% 第二种方式, 按照格式直接写文献信息, 使用第二种请把第一种全部注释
% 注意: 参考文献序号按所引文献在论文中出现的先后次序排列
% 第二种方式需要加下面三行命令才能把  "参考文献 " 加到目录

%\clearpage
%\phantomsection
%\addcontentsline{toc}{chapter}{参考文献} % 添加  "参考文献 " 到目录
%
%\begin{thebibliography}{99}
%\bibitem{Tadmor2012} Tadmor~E. A review of numerical methods for nonlinear partial differential
%  equations\allowbreak[J]. Bull. Amer. Math. Soc., 2012, 49(4): 507-554.
%
%\bibitem{LiLiu1997} 李荣华, 刘播. 微分方程数值解法\allowbreak[M]. 东南大学出版社, 1997.
%
%\bibitem{Adams2003} Adams~R~A, Fournier~J~J~F. Sobolev spaces\allowbreak[M]. Elsevier, 2003.
%
%\bibitem{TreWei2014}Trefethen~L~N, Weideman~J~A~C. The exponentially convergent trapezoidal rule\allowbreak[J]. SIAM Rev., 2014, 56(3): 385-458.
%
%\bibitem{Shen1994} Shen~J. Efficient spectral-Galerkin method I. Direct solvers of second- and fourth-order equations using Legendre polynomials\allowbreak[J]. SIAM J. Sci. Comput., 1994, 15(6): 1489-1505.
%
%\end{thebibliography}



%%%%%%%%%%%%%%%%%%%%%%%%%% 索引  %%%%%%%%%%%%%%%%%%%%%%%%%%%%%%%%%%%%%

\clearpage
\renewcommand\indexname{索~~引}
\phantomsection
\addcontentsline{toc}{chapter}{索~~引}
\printindex


%%%%%%%%%%%%%%%%%%%%%%%%%% 后记  %%%%%%%%%%%%%%%%%%%%%%%%%%%%%%%%%%%%%


\chapter{后~~记}

后记内容后记内容后记内容后记内容后记内容后记内容后记内容后记内容后记内容后记内容后记内容后记内容后记内容后记内容后记内容后记内容后记内容后记内容后记内容后记内容后记内容后记内容后记内容后记内容后记内容后记内容后记内容后记内容后记内容后记内容后记内容后记内容后记内容后记内容后记内容后记内容后记内容后记内容后记内容后记内容后记内容后记内容后记内容后记内容后记内容后记内容后记内容后记内容后记内容后记内容后记内容后记内容后记内容后记内容后记内容后记内容后记内容后记内容后记内容后记内容后记内容后记内容后记内容后记内容后记内容后记内容后记内容后记内容后记内容


\vspace{5ex}
\begin{flushright}
作~~者~~~~~~~~~

2021年12月~~~~~
\end{flushright}



\end{document}


